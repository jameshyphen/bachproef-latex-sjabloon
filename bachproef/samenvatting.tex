%%=============================================================================
%% Samenvatting
%%=============================================================================

% TODO: De "abstract" of samenvatting is een kernachtige (~ 1 blz. voor een
% thesis) synthese van het document.
%
% Deze aspecten moeten zeker aan bod komen:
% - Context: waarom is dit werk belangrijk?
% - Nood: waarom moest dit onderzocht worden?
% - Taak: wat heb je precies gedaan?
% - Object: wat staat in dit document geschreven?
% - Resultaat: wat was het resultaat?
% - Conclusie: wat is/zijn de belangrijkste conclusie(s)?
% - Perspectief: blijven er nog vragen open die in de toekomst nog kunnen
%    onderzocht worden? Wat is een mogelijk vervolg voor jouw onderzoek?
%
% LET OP! Een samenvatting is GEEN voorwoord!

%%---------- Nederlandse samenvatting -----------------------------------------
%
% TODO: Als je je bachelorproef in het Engels schrijft, moet je eerst een
% Nederlandse samenvatting invoegen. Haal daarvoor onderstaande code uit
% commentaar.
% Wie zijn bachelorproef in het Nederlands schrijft, kan dit negeren, de inhoud
% wordt niet in het document ingevoegd.

\IfLanguageName{english}{%
\selectlanguage{dutch}
\chapter*{Samenvatting}

Bedrijven die een middel- tot grote IT afdelingen hebben en groot-schalige infrastructuren moeten ondersteunen en beheren gaan vaak
het gebruik van Kubernetes aanschaffen. Door gebruik te maken van Kubernetes kunnen bedrijven heel makkelijk hun infrastructuur schaal veranderen
indien het nodig is. Dit is zeker handig voor snel-groeiende bedrijven zoals startups- of bedrijven die nieuwe projecten opstarten.
Het platform biedt de mogelijkheid om replica nodes aan te maken. Dit is handig om redundancy te hebben op de aangeboden software,
betere versie controle en nog meer. 

Wij onderzoeken wanneer een bedrijf van deze technologieën vaker gebruik van gaat maken
en met welke "Infrastructure as code" tools zoals Terraform, Ansible, Attune enz. Daarnaast gaan wij ook de performantie ervan
bestuderen bij verschillende scenario's en conclusies uit trekken.
\selectlanguage{english}
}{}

%%---------- Samenvatting -----------------------------------------------------
% De samenvatting in de hoofdtaal van het document

\chapter*{\IfLanguageName{dutch}{Samenvatting}{Abstract}}
Companies that have a middle- to big IT departments and need to support large-scale infrastructures will often make use
of Kubernetes to manage them. Through the use of Kubernetes, companies can easily scale up their infrasctuctures when it's necessary.
This is especially handy for fast-growing companies like startups- or for companies that are starting up new projects.
The platform offers the possibility to create replica nodes which is useful for redundancy of the software, version control and more.

Our research will cover when a company is more likely to use these technologies, and which "Infrastructure as code" tools they use, like Terraform, Ansible, Attune etc.
We will also conduct a series of tests on performance in different scenarios and draw our conclusions.

